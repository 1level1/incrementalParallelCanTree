\section{Introduction - Add formal definitions of FIS etc...  add more about the issue}
Mining of frequent items and association rules is a well known and studied field in Computer Science. The algorithms and solutions in this field can be roughly divided into two types - Apriori~\cite{agrawal1994fast} and tree based solutions~\cite{tsay2009fiut,leung2005cantree,tanbeer2009efficient} Each type has benefits and limitations such as simplicity, performance, memory consumptions and scaling. 
%We can also divide the requirements for frequent items mining, into use cases like:
%\begin{enumerate}
%\item Build and mine once.
%\item Build once and mine many different scenarios.
%\item Build and mine with support for incremental updates.
%\end{enumerate}


In this paper, we will describe an approach for dealing with an incrementally updated database, while avoiding candidate generation, and performing a single DB scan.

We will discuss previous related work, describe current technology and review implementation, usage and performance.

