\section{Introduction}
Mining of frequent items and association rules is a well known and studied field in Computer Science. The algorithms and solutions in this field can be roughly divided into two types - Apriori~\cite{agrawal1994fast} and tree based solutions~\cite{tsay2009fiut,leung2005cantree,tanbeer2009efficient} Each type has benefits and limitations such as simplicity, performance, memory consumptions and scaling. 
%We can also divide the requirements for frequent items mining, into use cases like:
%\begin{enumerate}
%\item Build and mine once.
%\item Build once and mine many different scenarios.
%\item Build and mine with support for incremental updates.
%\end{enumerate}


In this paper, we will describe an approach for dealing with an incrementally updated database, while avoiding candidate generation, and only a single DB scan.

We will discuss previous related work, describe current technology and review implementation, usage and performance.

For this article, we used 2 types of datasets:
\begin{enumerate}
\item Synthetic datasets of 100M, 10M and 1M transactions, and 2 magnitudes less of items, average length of 20 and 15 items. The datasets were generated using IBM Quest Synthetic Data Generator ~\cite{agrawal1994quest}.
\item The Kosarak dataset contains 990,000 transactions with 41,270 distinct items and an average transaction length of 8.09 items (click-stream data of a hungarian on-line news portal). This dataset was the largest used by ~\cite{tanbeer2009efficient}.
\end{enumerate}

