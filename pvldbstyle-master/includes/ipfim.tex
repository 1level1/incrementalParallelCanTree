\section{IPFIM - Incremental Parallel Frequent Itemsets Mining}
The implementation of this algorithm strongly depends on PFP~\cite{li2008pfp}. To support incremental tree updates, we are using a predefined comparison function to arrange the items insertion order, as used in CanTree~\cite{leung2005cantree}.
[TODO]
Do same as in PFP
\subsection{IPFIM Outline}
\begin{steps}
	\item Sharding:
	\end{steps}
\subsection{IPFIM structure}
\begin{steps}
	\item Define a comparison function for items: compare(item1,item2)->bool
	\item For each set of transactions in iteration:
		\begin{enumerate}
			\item For each Ti in current set of transaction, order by pre defined compare function.
			\item For each aj in Ti, replace aj with gi that aj belongs to its group
			\item For each gi, if it appears in Ti, find its right-most location in Ti, say L and output:
 <key'=gi; value'={Ti[0]…Ti[L]}>
 			\item Group by key' = gi
 			\item For each group gi, merge to existing tree if exists ELSE build new tree
 			\item For each group gi, mine the generated tree (filter items not in gi of length 1).
		\end{enumerate}
\end{steps}

\subsection{Correctness}
The correctness of the tree structure is driven from the correctness of mining CanTree, which preserves 2 properties:
\begin{properties}
\item \textit{The ordering of items is unaffected by the
changes in frequency caused by incremental updates.}
\item \textit{The frequency of a node in the CanTree is at
least as high as the sum of frequencies of its children.}
\end{properties}


